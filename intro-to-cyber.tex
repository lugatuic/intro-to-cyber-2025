\documentclass{beamer}

\usepackage{minted}
\usepackage{svg}

\usetheme{Boadilla}

\setminted{breaklines}

% Title
\title{Intro to Linux \& Cybersecurity: Hands-On Lab}
\date{\today}
\institute{Linux Users Group @ UIC\\ WiCyS @ UIC}

\begin{document}
\begin{frame}
	\titlepage
\end{frame}

\begin{frame}{Table of Contents}
	\tableofcontents[pausesections]
\end{frame}

\section{What's Linux/Cybersecurity?}
\begin{frame}{What's Linux?}
	\begin{columns}
		\begin{column}{0.5\textwidth}
			\textbf{Linux} is a free and open-source operating
			system kernel. Linux is part of a family of operating
			systems that bundle various pieces of software to form
			a complete OS, called \underline{Linux distros}.
		\end{column}
		\begin{column}{0.5\textwidth}
			\begin{figure}
				\centering
				\includesvg[width=0.9\textwidth]{tux.svg}
				\caption{Tux, the Linux mascot}
			\end{figure}
		\end{column}
	\end{columns}
\end{frame}

\begin{frame}{What's Cybersecurity?}
	stub
\end{frame}

\section{Hands-On Activity}
\begin{frame}{Table of Contents}
	\tableofcontents[currentsection]
\end{frame}

\subsection{Terminal and SSH}
\begin{frame}
	\begin{center}
		\Huge Open your terminal!
	\end{center}
\end{frame}

\begin{frame}{Connecting to a Server}
	\begin{Large}
		\textbf{Terminal} \\
	\end{Large}
	This is what your computer understands!
	\begin{figure}
		\centering
		\includegraphics[width=0.5\textwidth]{terminal.png}
		\caption{GNOME Terminal running Bash}
	\end{figure}
\end{frame}

\begin{frame}{Connecting to a Server}
	\begin{Large}
		\textbf{How to get to the terminal?} \\
	\end{Large}
	\begin{tabular}{|c|c|}
		\hline
		Windows & Open \texttt{Windows Powershell} \\
		\hline
		macOS and Linux & Open \texttt{Terminal} \\
		\hline
		iOS & Install \texttt{Terminus} \\
		\hline
		Android & Install \texttt{Termux} and \\
		& \texttt{sudo apt install openssh-client} \\
		\hline
	\end{tabular}
\end{frame}

\begin{frame}
	\begin{center}
		\begin{Huge}
			\texttt{ssh} \\
		\end{Huge}
		OpenSSH SSH client (remote login program)
	\end{center}
\end{frame}

\begin{frame}{SSH}
	\begin{Large}
		\textbf{Syntax} \\
	\end{Large}
	\texttt{ssh <username>@<server>}

	\vspace{0.3cm}

	\begin{Large}
		\textbf{Example} \\
	\end{Large}
	\inputminted{shell-session}{ssh.txt}
\end{frame}

\begin{frame}{Connecting to the Linux Week Server}
	\begin{Large}
		\textbf{Hostname} \\
	\end{Large}
	The server's hostname is \texttt{malware.cs.uic.edu}

	\vspace{0.3cm}

	\begin{Large}
		\textbf{Syntax} \\
	\end{Large}
	\texttt{ssh user<XX>@malware.cs.uic.edu} \\
	Where \texttt{<XX>} is a random number. \\
	The password is \texttt{uninstallwindows}.

	\vspace{0.3cm}

	\begin{Large}
		\textbf{Example} \\
	\end{Large}
	\inputminted{shell-session}{ssh-conn.txt}
\end{frame}

\section{Common Linux Utilities}
\subsection{Running Commands}
\begin{frame}{Table of Contents}
	\tableofcontents[currentsection]
\end{frame}

\begin{frame}{Intro to Coreutils}
	\begin{center}
		\Huge What are coreutils?
	\end{center}
\end{frame}

\begin{frame}{Common Examples}
	\begin{columns}
		\begin{column}{0.33\textwidth}
			\begin{figure}
				\centering
				\caption{cat}
				\includegraphics[width=0.7\textwidth]{cat.png}
				\includegraphics[width=0.7\textwidth]{rm.png}
				\caption{rm}
			\end{figure}
		\end{column}
		\begin{column}{0.33\textwidth}
			\begin{figure}
				\centering
				\caption{ls}
				\includegraphics[width=0.7\textwidth]{ls.png}
				\includegraphics[width=0.7\textwidth]{vi.png}
				\caption{vi}
			\end{figure}
		\end{column}
		\begin{column}{0.33\textwidth}
			\begin{figure}
				\centering
				\caption{mkdir}
				\includegraphics[width=0.7\textwidth]{mkdir.png}
				\includegraphics[width=0.7\textwidth]{cd.png}
				\caption{cd}
			\end{figure}
		\end{column}
	\end{columns}
\end{frame}

\begin{frame}{Structure of a Linux Command}
	\begin{Large}
		\textbf{Format} \\
	\end{Large}
	\texttt{\{command\} \{options/flags\} \{arguments\}}

	\vspace{0.3cm}

	\begin{Large}
		\textbf{Example} \\
	\end{Large}
	\texttt{rm -r oldStuff}

	\vspace{0.3cm}

	\begin{tabular}{|c|c|}
		\hline
		Command & \texttt{rm} \\
		\hline
		Flags & \texttt{-r} \\
		\hline
		Arguments & \texttt{oldStuff} \\
		\hline
	\end{tabular}
\end{frame}

\begin{frame}{Command Overview}
	\pause
	\begin{itemize}
		\item \texttt{ls}
		\begin{itemize}
			\item list directory contents
		\end{itemize}

		\pause

		\item \texttt{cd}
		\begin{itemize}
			\item change the shell working directory
		\end{itemize}

		\pause

		\item \texttt{mkdir}
		\begin{itemize}
			\item make directories
		\end{itemize}

		\pause

		\item \texttt{rm}
		\begin{itemize}
			\item remove files or directories
		\end{itemize}

		\pause

		\item \texttt{pwd}
		\begin{itemize}
			\item print name of current working directory
		\end{itemize}

		\pause

		\item \texttt{mv}
		\begin{itemize}
			\item move (rename) files
		\end{itemize}

		\pause

		\item \texttt{cp}
		\begin{itemize}
			\item copy files and directories
		\end{itemize}

		\pause

		\item \texttt{cat}
		\begin{itemize}
			\item concatanates/prints files
		\end{itemize}
	\end{itemize}
\end{frame}

\begin{frame}{I Need Help!}
	\pause
	\begin{center}
		\begin{Large}
			Use \texttt{man}! \\
		\end{Large}
		\pause
		Accesses reference manuals for \underline{all} commands on your system.
	\end{center}
	\pause
	\tiny\inputminted{shell-session}{man.txt}
\end{frame}

%\begin{frame}{ls}
%	\inputminted{shell-session}{ls.txt}
%\end{frame}
%
%\begin{frame}{cd}
%	\tiny\inputminted{shell-session}{cd.txt}
%\end{frame}
%
%\begin{frame}{mkdir}
%	\tiny\inputminted{shell-session}{mkdir.txt}
%\end{frame}
%
%\begin{frame}{rm}
%	\inputminted{shell-session}{rm.txt}
%\end{frame}
%
%\begin{frame}{pwd}
%	\inputminted{shell-session}{pwd.txt}
%\end{frame}
%
%\begin{frame}{mv}
%	\inputminted{shell-session}{mv.txt}
%\end{frame}
%
%\begin{frame}{cp}
%	\inputminted{shell-session}{cp.txt}
%\end{frame}
%
%\begin{frame}{cat}
%	\inputminted{shell-session}{cat.txt}
%\end{frame}

\section{Cybersecurity Commands}
\begin{frame}{Table of Contents}
	\tableofcontents[currentsection]
\end{frame}

\subsection{Permissions}
\begin{frame}{Permissions}
	In Linux, all files are \underline{owned} by a \textit{user} and a
	\textit{group}.
	\pause

	\begin{exampleblock}{ls output}
		\inputminted{shell-session}{lsal.txt}
	\end{exampleblock}
	\pause

	Permissions can be listed with \texttt{ls -al}.
\end{frame}

\begin{frame}{Permissions}
	Permissions are \textbf{important} because they safegard
	\textit{unauthorized users} from accessing data.
	\pause

	\vspace{0.3cm}

	\texttt{-rw-r--r-- 1 aether aether  6115622 Mar 25  2024 vibe\_2.mov}

	What does this mean?
\end{frame}

\begin{frame}{Permissions}
	We can change permissions of a file \textit{or} directory with \texttt{chmod}.
	\pause

	\tiny\inputminted{shell-session}{chmod.txt}
\end{frame}

\begin{frame}{Permissions}
	We can also change who \textbf{owns} a file with \texttt{chown}.
	\pause

	\tiny\inputminted{shell-session}{chown.txt}
	\pause

	\begin{center}
		\Large Try this!
	\end{center}
\end{frame}

\begin{frame}{Elevating Permissions}
	This didn't work because only the \texttt{root} user\footnote{the user
	with highest permissions in Linux} can change ownership to
	\texttt{root}.
	\pause

	\vspace{0.3cm}

	\textit{I cheated...}
	\pause

	\begin{exampleblock}{sudo}
		You can elevate to \texttt{root} permissions using
		\texttt{sudo}!
		\inputminted{shell-session}{sudo.txt}
	\end{exampleblock}
\end{frame}

\begin{frame}{Logs}
	\pause
	How is my system doing?
	\pause

	You can view logs in \texttt{/var/log}, which is a folder containing
	\underline{all logs}.
	\pause

	\begin{exampleblock}{\texttt{/var/log}}
		\tiny\inputminted{shell-session}{varlog.txt}
	\end{exampleblock}
\end{frame}

\begin{frame}{Closing Remarks}
	\begin{center}
		\Huge Thank you!
	\end{center}
\end{frame}

\begin{frame}{Closing Remarks}
	\begin{columns}
		\begin{column}{0.5\textwidth}
			\textbf{Officers}
			\begin{figure}
				\centering
				\includegraphics[width=0.60\textwidth]{officers.png}
			\end{figure}
		\end{column}
		\begin{column}{0.5\textwidth}
			The information in this presentation will be made
			available\footnotemark on our website!\\
			\url{https://lug.cs.uic.edu}
			
			\bigskip
			Join our Discord!

			\begin{figure}
				\centering
				\includesvg[width=0.5\textwidth]{lug-discord.svg}
				\caption{\url{https://discord.gg/NgxTR7PX5e}}
			\end{figure}
		\end{column}
	\end{columns}

	\footnotetext{sooner or later}
\end{frame}

\end{document}

% vim: set tw=80 ts=4 sw
